\documentclass[]{article}
\usepackage{lmodern}
\usepackage{amssymb,amsmath}
\usepackage{ifxetex,ifluatex}
\usepackage{fixltx2e} % provides \textsubscript
\ifnum 0\ifxetex 1\fi\ifluatex 1\fi=0 % if pdftex
  \usepackage[T1]{fontenc}
  \usepackage[utf8]{inputenc}
\else % if luatex or xelatex
  \ifxetex
    \usepackage{mathspec}
  \else
    \usepackage{fontspec}
  \fi
  \defaultfontfeatures{Ligatures=TeX,Scale=MatchLowercase}
\fi
% use upquote if available, for straight quotes in verbatim environments
\IfFileExists{upquote.sty}{\usepackage{upquote}}{}
% use microtype if available
\IfFileExists{microtype.sty}{%
\usepackage{microtype}
\UseMicrotypeSet[protrusion]{basicmath} % disable protrusion for tt fonts
}{}
\usepackage[margin=1in]{geometry}
\usepackage{hyperref}
\hypersetup{unicode=true,
            pdftitle={Maximum Likehood Estimator},
            pdfborder={0 0 0},
            breaklinks=true}
\urlstyle{same}  % don't use monospace font for urls
\usepackage{color}
\usepackage{fancyvrb}
\newcommand{\VerbBar}{|}
\newcommand{\VERB}{\Verb[commandchars=\\\{\}]}
\DefineVerbatimEnvironment{Highlighting}{Verbatim}{commandchars=\\\{\}}
% Add ',fontsize=\small' for more characters per line
\usepackage{framed}
\definecolor{shadecolor}{RGB}{248,248,248}
\newenvironment{Shaded}{\begin{snugshade}}{\end{snugshade}}
\newcommand{\AlertTok}[1]{\textcolor[rgb]{0.94,0.16,0.16}{#1}}
\newcommand{\AnnotationTok}[1]{\textcolor[rgb]{0.56,0.35,0.01}{\textbf{\textit{#1}}}}
\newcommand{\AttributeTok}[1]{\textcolor[rgb]{0.77,0.63,0.00}{#1}}
\newcommand{\BaseNTok}[1]{\textcolor[rgb]{0.00,0.00,0.81}{#1}}
\newcommand{\BuiltInTok}[1]{#1}
\newcommand{\CharTok}[1]{\textcolor[rgb]{0.31,0.60,0.02}{#1}}
\newcommand{\CommentTok}[1]{\textcolor[rgb]{0.56,0.35,0.01}{\textit{#1}}}
\newcommand{\CommentVarTok}[1]{\textcolor[rgb]{0.56,0.35,0.01}{\textbf{\textit{#1}}}}
\newcommand{\ConstantTok}[1]{\textcolor[rgb]{0.00,0.00,0.00}{#1}}
\newcommand{\ControlFlowTok}[1]{\textcolor[rgb]{0.13,0.29,0.53}{\textbf{#1}}}
\newcommand{\DataTypeTok}[1]{\textcolor[rgb]{0.13,0.29,0.53}{#1}}
\newcommand{\DecValTok}[1]{\textcolor[rgb]{0.00,0.00,0.81}{#1}}
\newcommand{\DocumentationTok}[1]{\textcolor[rgb]{0.56,0.35,0.01}{\textbf{\textit{#1}}}}
\newcommand{\ErrorTok}[1]{\textcolor[rgb]{0.64,0.00,0.00}{\textbf{#1}}}
\newcommand{\ExtensionTok}[1]{#1}
\newcommand{\FloatTok}[1]{\textcolor[rgb]{0.00,0.00,0.81}{#1}}
\newcommand{\FunctionTok}[1]{\textcolor[rgb]{0.00,0.00,0.00}{#1}}
\newcommand{\ImportTok}[1]{#1}
\newcommand{\InformationTok}[1]{\textcolor[rgb]{0.56,0.35,0.01}{\textbf{\textit{#1}}}}
\newcommand{\KeywordTok}[1]{\textcolor[rgb]{0.13,0.29,0.53}{\textbf{#1}}}
\newcommand{\NormalTok}[1]{#1}
\newcommand{\OperatorTok}[1]{\textcolor[rgb]{0.81,0.36,0.00}{\textbf{#1}}}
\newcommand{\OtherTok}[1]{\textcolor[rgb]{0.56,0.35,0.01}{#1}}
\newcommand{\PreprocessorTok}[1]{\textcolor[rgb]{0.56,0.35,0.01}{\textit{#1}}}
\newcommand{\RegionMarkerTok}[1]{#1}
\newcommand{\SpecialCharTok}[1]{\textcolor[rgb]{0.00,0.00,0.00}{#1}}
\newcommand{\SpecialStringTok}[1]{\textcolor[rgb]{0.31,0.60,0.02}{#1}}
\newcommand{\StringTok}[1]{\textcolor[rgb]{0.31,0.60,0.02}{#1}}
\newcommand{\VariableTok}[1]{\textcolor[rgb]{0.00,0.00,0.00}{#1}}
\newcommand{\VerbatimStringTok}[1]{\textcolor[rgb]{0.31,0.60,0.02}{#1}}
\newcommand{\WarningTok}[1]{\textcolor[rgb]{0.56,0.35,0.01}{\textbf{\textit{#1}}}}
\usepackage{graphicx,grffile}
\makeatletter
\def\maxwidth{\ifdim\Gin@nat@width>\linewidth\linewidth\else\Gin@nat@width\fi}
\def\maxheight{\ifdim\Gin@nat@height>\textheight\textheight\else\Gin@nat@height\fi}
\makeatother
% Scale images if necessary, so that they will not overflow the page
% margins by default, and it is still possible to overwrite the defaults
% using explicit options in \includegraphics[width, height, ...]{}
\setkeys{Gin}{width=\maxwidth,height=\maxheight,keepaspectratio}
\IfFileExists{parskip.sty}{%
\usepackage{parskip}
}{% else
\setlength{\parindent}{0pt}
\setlength{\parskip}{6pt plus 2pt minus 1pt}
}
\setlength{\emergencystretch}{3em}  % prevent overfull lines
\providecommand{\tightlist}{%
  \setlength{\itemsep}{0pt}\setlength{\parskip}{0pt}}
\setcounter{secnumdepth}{0}
% Redefines (sub)paragraphs to behave more like sections
\ifx\paragraph\undefined\else
\let\oldparagraph\paragraph
\renewcommand{\paragraph}[1]{\oldparagraph{#1}\mbox{}}
\fi
\ifx\subparagraph\undefined\else
\let\oldsubparagraph\subparagraph
\renewcommand{\subparagraph}[1]{\oldsubparagraph{#1}\mbox{}}
\fi

%%% Use protect on footnotes to avoid problems with footnotes in titles
\let\rmarkdownfootnote\footnote%
\def\footnote{\protect\rmarkdownfootnote}

%%% Change title format to be more compact
\usepackage{titling}

% Create subtitle command for use in maketitle
\newcommand{\subtitle}[1]{
  \posttitle{
    \begin{center}\large#1\end{center}
    }
}

\setlength{\droptitle}{-2em}

  \title{Maximum Likehood Estimator}
    \pretitle{\vspace{\droptitle}\centering\huge}
  \posttitle{\par}
    \author{}
    \preauthor{}\postauthor{}
    \date{}
    \predate{}\postdate{}
  

\begin{document}
\maketitle

\hypertarget{introduction}{%
\section{Introduction}\label{introduction}}

maximum likelihood estimation is one common method for estimating
paremater in a parametric model, just like method of moments. We
assume\(X_1, X_2..., X_n\) be IID with \emph{PDF} \(f(x;\theta)\),
define the likelihood function as \[
\mathcal{L}(\theta)=\Pi_{i=1}^nf(X_i;\theta)
\] And the log likelihood function is defined by
\(\mathcal{l}_n(\theta)=log \mathcal{L}_n(\theta)=\sum_{i=1}^n log f(X_i; \theta)\).

The \emph{maximum likelihood estimator} MLE, denoted by
\(\hat{\theta}_n\), is the value of \(\theta\) that maximizes
\(\mathcal{L}_n(\theta)\)

\hypertarget{derivation}{%
\section{Derivation}\label{derivation}}

\hypertarget{point-mass-at-a}{%
\subsubsection{\texorpdfstring{Point mass at
\(a\)}{Point mass at a}}\label{point-mass-at-a}}

Notice the probability mass function for point mass distribution is
\(f(x)=1\) in \(x=a\) and 0 elsewhere. In fact, \(f\) have only two
values:1 and 0, so does the \(\mathcal{L}(\theta)\) which is the product
of many \(f\)s. We choose the MLE for point mass distribution to be mode
among \(X_n\).

\hypertarget{bernoulli}{%
\subsubsection{Bernoulli}\label{bernoulli}}

The probability function is \(f(x;p)=p^x(1-p)^{1-x}\) for \(x=0,1\) ,
the unknown parameter is \(p\). \[
\mathcal{L}_n(p)=\prod_{i=1}^nf(X_i; p)=\prod_{i=1}^n p^{X_i}(1-p)^{1-X_i}=p^S(1-p)^{n-S}
\] where \(S=\sum_i X_i\). Hence, \[
\mathcal{l}_n(p)=S log p+ (n-S)log(1-p)
\] Take the derivation and set it equal to 0 gave us
\(\hat{p}_n=\frac{S}{n}\)

\hypertarget{binomialnp}{%
\subsubsection{Binomial(N,p)}\label{binomialnp}}

\[
\mathcal{L}(p) = \prod_i^n(f(y_i)) =  \prod_i^n \left[ {{N}\choose{y_i}}p^{y_i} (1-p)^{N- yi} \right] = \left[ \prod_i^n {{N}\choose{y_i}} \right] p^{\sum_1^n y_i} (1-p)^{nN - \sum_1^n{y_i}} \,.\]

\[\hat{p} = \arg\max_p \left[ \left[ \prod_i^n {{N}\choose{y_i}} \right] p^{\sum_1^n y_i} (1-p)^{nN - \sum_1^n{y_i}}\right]\]

\[
\begin{array}{rcl} \hat{p} & = & \displaystyle\arg\max_p\ \  \ln\left[ \left[ \prod_i^n {{N}\choose{y_i}} \right] p^{\sum_1^n y_i} (1-p)^{nN - \sum_1^n{y_i}}\right] \\ & = & \displaystyle \arg\limits\max_p  \sum_{i=1}^n\left[ \ln {{N}\choose{y_i}} + \left( \sum_{i=1}^n y_i \right)\ln(p) + \left( nN - \sum_{i=1}^n y_i \right)\ln(1-p) \right] \end{array}
\]

\[
0 + \frac{\left( \sum_{i=1}^n y_i \right)}{\hat{p}} - \frac{\left(n N - \sum_{i=1}^n y_i \right)}{1-\hat{p}} = 0 \,.
\] \[\hat{p} = \frac{\sum_{i=1}^n y_i}{nN} = \frac{\bar{y}}{N} \]

\hypertarget{geometricp}{%
\subsubsection{Geometric(p)}\label{geometricp}}

\(P(X=k)=p(1-p)^k\) where \(k\ge1\) \[
\mathcal{L}\left(p \right)={\left(1-p \right)}^{{x}_{1}-1}p {\left(1-p \right)}^{{x}_{2}-1}p...{\left(1-p \right)}^{{x}_{n}-1}p ={p}^{n}{\left(1-p \right)}^{\sum_{1}^{n}{x}_{i}-n}
\]
\[ln\mathcal{L}(p)=\sum_{i=1}^{n}{x}_{i}ln(p)+\left(n-\sum_{i=1}^{n}{x}_{i} \right)ln\left(1-p \right)\]

\[
\frac{dln\mathcal{L}(p)}{dp}=\frac{1}{p}\sum_{i=1}^{n}{x}_{i}+\frac{1}{1-p}\left(n-\sum_{i=1}^{n}{x}_{i} \right)=0
\] \[\hat{p}=\frac{n}{\left(\sum_{1}^{n}{x}_{i} \right)}\]

\hypertarget{poisson-distributionlambda}{%
\subsubsection{\texorpdfstring{Poisson
Distribution(\(\lambda\))}{Poisson Distribution(\textbackslash{}lambda)}}\label{poisson-distributionlambda}}

\[
\mathcal{L}(\lambda)=\prod_{i=1}^{n}\frac{{\lambda}^{{x}_{i}}{e}^{-\lambda}}{{x}_{i}!} = {e}^{-n\lambda} \frac{{\lambda}^{\sum_{1}^{n}{x}_{i}}}{\prod_{i=1}^{n}{x}_{i}!}
\] \[
lnL(\lambda)=-n\lambda+\sum_{1}^{n}{x}_{i}ln(\lambda)-ln\left(\prod_{i=1}^{n}{x}_{i}!\right)
\]
\[\frac{dlnL(\lambda)}{dp}=-n+   \sum_{1}^{n} \frac{ x_i } {\lambda} \]

\[\hat{\lambda}=\frac{\sum_{i=1}^{n}{x}_{i}}{n}\]

\hypertarget{uniformab}{%
\subsubsection{Uniform(a,b)}\label{uniformab}}

\[
\mathcal{L}_n(a, b)=\frac{1}{(b-a)^n}
\] if all \(a \le X_1, X_2, ..., X_n \le b\) else \(\mathcal{L}=0\) To
maximize, we need to minize \(b-a\), the boundary condition is to choose
\(\hat{a}=min_i(X_i)\) and \(\hat{b}=max_i(X_i)\)

\hypertarget{normalmu-sigma2}{%
\subsubsection{\texorpdfstring{Normal(\(\mu\),
\(\sigma^2\))}{Normal(\textbackslash{}mu, \textbackslash{}sigma\^{}2)}}\label{normalmu-sigma2}}

\[
\mathcal{L}_n(\mu, \sigma) =\prod_i \frac{1}{\sigma} exp\{- \frac{1}{2 \sigma^2} (X_i-\mu)^2 \} \\
                          =\sigma^{-n}exp\{ -\frac{1}{2\sigma^2}\sum_i(X_i-\mu)^2 \}\\
                          =\sigma^{-n}exp\{ -\frac{nS^2}{2\sigma^2}\} exp\{- \frac{n(\bar{X} -\mu)^2}{2\sigma^2} \}
\] where \(\bar{X}=n^{-1}\sum_i X_i\),
\(S^2=n^{-1}\sum_i (X_i - \bar{X})^2\)

\[
\mathcal{l}(\mu, \sigma)=-n log\sigma - \frac{nS^2}{2\sigma^2}- \frac{n(\bar{X}-\mu )^2}{2\sigma^2}
\] Solving eqations
\(\frac{ \partial \mathcal{l}(\mu, \sigma) }{\partial \mu} =0\) and
\(\frac{ \partial \mathcal{l}(\mu, \sigma)} {\partial \mathcal{l} \sigma}\)
we conclude that \(\hat{\mu}=\bar{X}\) and \(\hat{\sigma}=S\)

\hypertarget{exponentialbeta}{%
\subsubsection{\texorpdfstring{Exponential(\(\beta\))}{Exponential(\textbackslash{}beta)}}\label{exponentialbeta}}

\[
\mathcal{L}(\beta)=\mathcal{L}\left(\beta;{X}_{1},{X}_{2}...{X}_{n} \right)=\left(\frac{1}{\beta}{e}^{\frac{{-X}_{1}}{\beta}}\right)\left(\frac{1}{\theta}{e}^{\frac{{-X}_{2}}{\beta}}\right)...\left(\frac{1}{\theta}{e}^{\frac{{-X}_{n}}{\beta}} \right)=\frac{1}{{\beta}^{n}}exp\left(\frac{-\sum_{1}^{n}{X}_{i}}{\beta} \right)
\] \[
ln \mathcal{L}\left(\beta\right)=-n ln\left(\beta\right) -\frac{1}{\beta}\sum_{1}^{n}{X}_{i}, 0<\beta<\infty\]
\[
\frac{d\left[ln \mathcal{L}\left(\theta\right) \right]}{d\beta}=\frac{-n }{\beta} +\frac{1}{{\beta}^{2}}\sum_{1}^{n}{X}_{i}=0
\]

\[
\hat{\beta}=\frac{\sum_{1}^{n}{X}_{i}}{n}
\]

\hypertarget{gammaalpha-beta}{%
\subsubsection{\texorpdfstring{Gamma(\(\alpha, \beta\))}{Gamma(\textbackslash{}alpha, \textbackslash{}beta)}}\label{gammaalpha-beta}}

\[
\mathcal{L}(\alpha, \beta)=(\frac{1}{\beta^\alpha \Gamma(\alpha)})^n exp^{-\frac{\sum_iX_i}{\beta}} \prod_{i=1}^n X_i^{\alpha-1}
\]

\[
\mathcal{l}_n(\alpha, \beta)=
-n \alpha log \beta -n log\Gamma(\alpha) -\frac{\sum_i X_i}{\beta}
+\sum_{i=1}^n (\alpha -1) log X_i
\] Solve
\(\frac{\partial \mathcal{l}_n(\alpha, \beta)}{\partial \beta}=0\) and
\(\frac{\partial l_n(\alpha, \beta)}{ \partial \alpha}=0\) yield: \[
\hat{\beta}=\frac{\sum_i X_i}{n\alpha}
\]

and \[
\sum_{i=1}^n log X_i = n og \frac{\sum X_i}{n \alpha} + n \psi(\alpha) 
\]

where \[\psi(\alpha)=\frac{d log \Gamma(\alpha) }{d \alpha}\], is called
Digamma Function However, no closed form exists for \(\hat{\alpha}\),
only numerical solution exist.

we calculate the numerical estimation using Newton-Raphson
\href{http://bioops.info/2015/01/gamma-mme-mle/}{Refernce} \[
l_n'(\alpha)=n log(\frac{\alpha}{\bar{X}})-n\frac{\Gamma'(\alpha)}{\Gamma(\alpha)}+\sum_{i=1}^nlog X_i
\] \[
l_n''(\alpha)=\frac{n}{\alpha}-n(\frac{\Gamma'(\alpha)}{\Gamma(\alpha)})'
\] \[
\hat{\alpha}^{(k)}=\hat{\alpha}^{(k-1)}- \frac{l_n'(\hat{\alpha} ^{(k-1)})} {l_n''(\hat{\alpha} ^{(k-1)})}
\\
\hat{\beta}=\frac{\bar{X}}{\hat{\alpha}}
\] \#\#\# Beta(\(\alpha, \beta\)) \[
\mathcal{L}_n(\alpha, \beta)=(\frac{\Gamma(\alpha +\beta)}{\Gamma(\alpha)\Gamma(\beta)})^n \prod_{i=1}^n(X_i)^{\alpha-1}
\prod_{i=1}^n(1-X_i)^{\beta-1}
\] \[
\mathcal{l}_n(\alpha, \beta)=nlog( \Gamma (\alpha +\beta) ) -n log( \Gamma(\beta)) +(\alpha-1)\sum_{i=1}^n log(X_i)+(\beta-1) \sum_{i=1}^n log(1-X_i)
\] solve
\(\frac{\partial \mathcal{l}_n(\alpha, \beta)}{ \partial \alpha}\) and
\(\frac{ \partial \mathcal{l}_n(\alpha, \beta)}{\partial \beta}\) yield
no analytical solution. However, numerical solution via Newton-Raphson
mehod does exist.
\href{https://scholarsarchive.byu.edu/cgi/viewcontent.cgi?article=2613\&context=etd}{For
reference, check page 27 on this paper}

We set \(\hat{\theta}=(\hat{\alpha}, \hat{ \beta})\) iteratively: \$\$
\hat{\theta}\emph{\{i+1\}=\hat{\theta}}\{i\}-
G\textsuperscript{\{-1\}g\textbackslash{} g={[}g1,g2{]}\textbackslash{}
g1=\psi(\hat{\alpha})-\psi(\hat{\alpha}+\hat{\beta})-\frac{1}{n}\sum\emph{\{i=1\}\^{}n
log(X\_i)\textbackslash{}
g2=\psi(\hat{\beta})-\psi(\hat{\alpha}+\hat{\beta})-\frac{1}{n}\sum}\{i=1\}}n
log(1-X\_i)\textbackslash{} G=

\begin{bmatrix}
\frac{dg1}{d\alpha} & \frac{dg2}{d\beta} \\
\frac{dg2}{d\alpha} & \frac{dg2}{d\beta}

\end{bmatrix}

\textbackslash{} \frac{dg1}{d\alpha}=\psi`(\alpha) -
\psi'(\alpha +\beta) \textbackslash{}
\frac{dg1}{d\beta}=\frac{dg2}{d\beta}=-\psi`(\alpha +
\beta)\textbackslash{}
\frac{dg2}{d\beta}=\psi'(\beta)-\psi'(\alpha+\beta) \$\$

\hypertarget{student-t-distribution}{%
\subsubsection{Student t distribution}\label{student-t-distribution}}

\[
\mathcal{L}_n(v)=(\frac{\Gamma( \frac{\nu+1}{2} )}{\Gamma{\frac{ \nu}{ 2}} })^n \frac{1}
{
\prod_{i=1}^n(1+\frac{X_i^2}{\nu})^{ \frac{\nu+1} {2} }
} 
\] \[
l_n(\nu)=n log(\Gamma(\frac{\nu+1}{2}))-n log( \Gamma( \frac{\nu}{2} ) )-\sum_{i=1}^n \frac{\nu+1}{2}log(1+ \frac{X_i^2}{\nu} )
\]
\[\frac{\partial l_n(\nu)}{\partial\nu}= \frac{\nu+1}{2}\sum_{i=1}^n \frac{X_i^2}{\nu^2}-\frac{1}{2}\sum_{i=1}^n(\frac{X_i^2}{\nu}+1)+\frac{n \psi( \frac{\nu+1}{2})}{2}- \frac{n \psi(\frac{\nu}{2})}{2}=0
\] We cannot get closed-form solution.

\hypertarget{mathcalchi_p2-chi-squared-distriubtion}{%
\subsubsection{\texorpdfstring{\(\mathcal{\chi}_p^2\) Chi-Squared
Distriubtion}{\textbackslash{}mathcal\{\textbackslash{}chi\}\_p\^{}2 Chi-Squared Distriubtion}}\label{mathcalchi_p2-chi-squared-distriubtion}}

\[
\mathcal{L}_n(p)=\frac{1}{(\Gamma(p/2) 2^{p/2})^n}e^{\frac{-\sum_{i=1}^nX_i}{2}} \prod_{i=1}^n X_i^{p/2-1}
\]

\[
l_n(p)=-n log(\Gamma(p/2))- \frac{pn}{2} log2-\sum_{i=1}^n\frac{X_i}{2}+(\frac{p}{2}-1)\sum_{i=1}^n log X_i
\]

\[
\frac{\partial l_n(p)}{\partial p}=
- \frac{n}{2}\psi(p/2)-\frac{n}{2} log2+\frac{1}{2}\sum_{i=1}^n log X_i=0
\]

\[
\hat{p}=2 \psi^{-1}(\frac{\sum_{i=1}^n log X_i}{n} - log 2)
\] But \href{http://mathworld.wolfram.com/DigammaFunction.html}{in
fact}, digamma function is not monotone. Therefore it doesn't have an
inverse function at all. So the formula above is not right! In other
words, the student t distributions doesn't have a closed-form expression
for MLE.

\hypertarget{program}{%
\section{Program}\label{program}}

\begin{Shaded}
\begin{Highlighting}[]
\NormalTok{MLE_Beta<-}\ControlFlowTok{function}\NormalTok{(x, }\DataTypeTok{iteration=}\DecValTok{100}\NormalTok{)\{}
\NormalTok{  mu=}\KeywordTok{mean}\NormalTok{(x)}
\NormalTok{  s2<-}\KeywordTok{var}\NormalTok{(x)}
\NormalTok{  theta<-}\KeywordTok{c}\NormalTok{(mu}\OperatorTok{*}\NormalTok{(}\OperatorTok{-}\DecValTok{1}\OperatorTok{+}\StringTok{ }\NormalTok{mu}\OperatorTok{*}\NormalTok{(}\DecValTok{1}\OperatorTok{-}\NormalTok{mu)}\OperatorTok{/}\NormalTok{s2),}
\NormalTok{    (}\DecValTok{1}\OperatorTok{-}\NormalTok{mu)}\OperatorTok{*}\NormalTok{(}\OperatorTok{-}\DecValTok{1}\OperatorTok{+}\StringTok{ }\NormalTok{mu}\OperatorTok{*}\NormalTok{(}\DecValTok{1}\OperatorTok{-}\NormalTok{mu)}\OperatorTok{/}\NormalTok{s2)}
\NormalTok{  )}
\NormalTok{  G<-}\KeywordTok{matrix}\NormalTok{(}\DecValTok{0}\NormalTok{, }\DataTypeTok{ncol=}\DecValTok{2}\NormalTok{, }\DataTypeTok{nrow=}\DecValTok{2}\NormalTok{)}
\NormalTok{  g<-}\KeywordTok{matrix}\NormalTok{(}\DecValTok{0}\NormalTok{, }\DataTypeTok{nrow=}\DecValTok{2}\NormalTok{, }\DataTypeTok{ncol=}\DecValTok{1}\NormalTok{)}
  \ControlFlowTok{for}\NormalTok{(i }\ControlFlowTok{in} \DecValTok{1}\OperatorTok{:}\NormalTok{iteration)\{}
\NormalTok{    g[}\DecValTok{1}\NormalTok{,}\DecValTok{1}\NormalTok{]=}\KeywordTok{digamma}\NormalTok{(theta[}\DecValTok{1}\NormalTok{])}\OperatorTok{-}\KeywordTok{digamma}\NormalTok{(theta[}\DecValTok{1}\NormalTok{] }\OperatorTok{+}\StringTok{ }\NormalTok{theta[}\DecValTok{2}\NormalTok{])}\OperatorTok{-}\KeywordTok{sum}\NormalTok{(}\KeywordTok{log}\NormalTok{(x))}\OperatorTok{/}\KeywordTok{length}\NormalTok{(x)}
\NormalTok{    g[}\DecValTok{2}\NormalTok{,}\DecValTok{1}\NormalTok{]=}\KeywordTok{digamma}\NormalTok{(theta[}\DecValTok{2}\NormalTok{])}\OperatorTok{-}\KeywordTok{digamma}\NormalTok{(theta[}\DecValTok{1}\NormalTok{] }\OperatorTok{+}\StringTok{ }\NormalTok{theta[}\DecValTok{2}\NormalTok{])}\OperatorTok{-}\KeywordTok{sum}\NormalTok{(}\KeywordTok{log}\NormalTok{(}\DecValTok{1}\OperatorTok{-}\NormalTok{x))}\OperatorTok{/}\KeywordTok{length}\NormalTok{(x)}
\NormalTok{    G[}\DecValTok{1}\NormalTok{,}\DecValTok{1}\NormalTok{]=}\KeywordTok{trigamma}\NormalTok{(theta[}\DecValTok{1}\NormalTok{])}\OperatorTok{-}\KeywordTok{trigamma}\NormalTok{(theta[}\DecValTok{1}\NormalTok{]}\OperatorTok{+}\NormalTok{theta[}\DecValTok{2}\NormalTok{])}
\NormalTok{    G[}\DecValTok{1}\NormalTok{,}\DecValTok{2}\NormalTok{]=}\OperatorTok{-}\KeywordTok{trigamma}\NormalTok{(theta[}\DecValTok{1}\NormalTok{]}\OperatorTok{+}\NormalTok{theta[}\DecValTok{2}\NormalTok{])}
\NormalTok{    G[}\DecValTok{2}\NormalTok{,}\DecValTok{1}\NormalTok{]=}\OperatorTok{-}\KeywordTok{trigamma}\NormalTok{(theta[}\DecValTok{1}\NormalTok{]}\OperatorTok{+}\NormalTok{theta[}\DecValTok{2}\NormalTok{])}
\NormalTok{    G[}\DecValTok{2}\NormalTok{,}\DecValTok{2}\NormalTok{]=}\KeywordTok{trigamma}\NormalTok{(theta[}\DecValTok{2}\NormalTok{])}\OperatorTok{-}\KeywordTok{trigamma}\NormalTok{(theta[}\DecValTok{1}\NormalTok{]}\OperatorTok{+}\NormalTok{theta[}\DecValTok{2}\NormalTok{])}
\NormalTok{    theta<-theta}\OperatorTok{-}\KeywordTok{solve}\NormalTok{(G) }\OperatorTok\StringTok{ }\NormalTok{g }
\NormalTok{  \}}
  \KeywordTok{return}\NormalTok{( }\KeywordTok{list}\NormalTok{(}\DataTypeTok{shape1=}\NormalTok{theta[}\DecValTok{1}\NormalTok{], }\DataTypeTok{shape2=}\NormalTok{theta[}\DecValTok{2}\NormalTok{] ) )}
\NormalTok{\}}

\NormalTok{MLE_Gamma<-}\ControlFlowTok{function}\NormalTok{(x, }\DataTypeTok{iteratin=}\DecValTok{100}\NormalTok{)\{}
\NormalTok{  n<-}\KeywordTok{length}\NormalTok{(x)}
\NormalTok{  mean_x<-}\KeywordTok{mean}\NormalTok{(x)}
\NormalTok{  alpha<-n}\OperatorTok{*}\NormalTok{(mean_x}\OperatorTok{^}\DecValTok{2}\NormalTok{)}\OperatorTok{/}\KeywordTok{sum}\NormalTok{((x}\OperatorTok{-}\NormalTok{mean_x)}\OperatorTok{^}\DecValTok{2}\NormalTok{)}
\NormalTok{  beta<-}\KeywordTok{sum}\NormalTok{((x}\OperatorTok{-}\NormalTok{mean_x)}\OperatorTok{^}\DecValTok{2}\NormalTok{)}\OperatorTok{/}\NormalTok{n}\OperatorTok{/}\NormalTok{mean_x}
  \ControlFlowTok{for}\NormalTok{(i }\ControlFlowTok{in} \DecValTok{1}\OperatorTok{:}\NormalTok{iteratin)\{}
    \CommentTok{#first derivative of alpha_k-1}
\NormalTok{    der1<-n}\OperatorTok{*}\KeywordTok{log}\NormalTok{(alpha}\OperatorTok{/}\NormalTok{mean_x)}\OperatorTok{-}\NormalTok{n}\OperatorTok{*}\KeywordTok{digamma}\NormalTok{(alpha)}\OperatorTok{+}\KeywordTok{sum}\NormalTok{(}\KeywordTok{log}\NormalTok{(x))}
    \CommentTok{#second derivative of alpha_k-1}
\NormalTok{    der2<-n}\OperatorTok{/}\NormalTok{alpha}\OperatorTok{-}\NormalTok{n}\OperatorTok{*}\KeywordTok{trigamma}\NormalTok{(alpha)}
    \CommentTok{#calculate next alpha}
\NormalTok{    alpha<-alpha}\OperatorTok{-}\NormalTok{der1}\OperatorTok{/}\NormalTok{der2}
\NormalTok{    beta<-mean_x}\OperatorTok{/}\NormalTok{alpha}
\NormalTok{  \}}

  \KeywordTok{return}\NormalTok{(}\KeywordTok{list}\NormalTok{(}\DataTypeTok{shape=}\NormalTok{alpha, }\DataTypeTok{scale=}\NormalTok{beta))}
\NormalTok{\}}

\NormalTok{MLE<-}\ControlFlowTok{function}\NormalTok{(x, type, }\DataTypeTok{bino=}\DecValTok{5}\NormalTok{)\{}
  \ControlFlowTok{if}\NormalTok{(type}\OperatorTok{==}\StringTok{"PointMass"}\NormalTok{) \{ ux <-}\StringTok{ }\KeywordTok{unique}\NormalTok{(x) ; }\KeywordTok{return}\NormalTok{(  ux[}\KeywordTok{which.max}\NormalTok{(}\KeywordTok{tabulate}\NormalTok{(}\KeywordTok{match}\NormalTok{(x, ux)))]) \}}
  \ControlFlowTok{if}\NormalTok{ (type}\OperatorTok{==}\StringTok{"Bernoulli"}\NormalTok{) }\KeywordTok{return}\NormalTok{(  }\KeywordTok{list}\NormalTok{(}\DataTypeTok{size=}\DecValTok{1}\NormalTok{, }\DataTypeTok{prob=}\KeywordTok{sum}\NormalTok{(x)}\OperatorTok{/}\KeywordTok{length}\NormalTok{(x)) )}
  \ControlFlowTok{if}\NormalTok{(type}\OperatorTok{==}\StringTok{"Binomial"}\NormalTok{) }\KeywordTok{return}\NormalTok{(  }\KeywordTok{list}\NormalTok{(}\DataTypeTok{size=}\NormalTok{bino, }\DataTypeTok{prob=}\KeywordTok{mean}\NormalTok{(x)}\OperatorTok{/}\NormalTok{bino) )}
  \ControlFlowTok{if}\NormalTok{(type}\OperatorTok{==}\StringTok{"Geometric"}\NormalTok{) }\KeywordTok{return}\NormalTok{( }\KeywordTok{list}\NormalTok{(}\DataTypeTok{prob=}\KeywordTok{length}\NormalTok{(x)}\OperatorTok{/}\KeywordTok{mean}\NormalTok{(x) ))}
  \ControlFlowTok{if}\NormalTok{(type}\OperatorTok{==}\StringTok{"Poisson"}\NormalTok{) }\KeywordTok{return}\NormalTok{( }\KeywordTok{list}\NormalTok{( }\DataTypeTok{lambda=}\KeywordTok{mean}\NormalTok{(x) ) )}
  \ControlFlowTok{if}\NormalTok{(type}\OperatorTok{==}\StringTok{"Uniform"}\NormalTok{) }\KeywordTok{return}\NormalTok{( }\KeywordTok{list}\NormalTok{(}\DataTypeTok{min=}\KeywordTok{min}\NormalTok{(x), }\DataTypeTok{max=}\KeywordTok{max}\NormalTok{(x) ) )}
  \ControlFlowTok{if}\NormalTok{(type}\OperatorTok{==}\StringTok{"Normal"}\NormalTok{ ) }\KeywordTok{return}\NormalTok{( }\KeywordTok{list}\NormalTok{(}\DataTypeTok{mean=}\KeywordTok{mean}\NormalTok{(x), }\DataTypeTok{sd=}\KeywordTok{sqrt}\NormalTok{(  }\KeywordTok{sum}\NormalTok{( (x}\OperatorTok{-}\KeywordTok{mean}\NormalTok{(x) )}\OperatorTok{^}\DecValTok{2}\NormalTok{)}\OperatorTok{/}\StringTok{ }\KeywordTok{length}\NormalTok{(x)  )))}
  \ControlFlowTok{if}\NormalTok{(type}\OperatorTok{==}\StringTok{"Exponential"}\NormalTok{) }\KeywordTok{return}\NormalTok{( }\KeywordTok{list}\NormalTok{(}\DataTypeTok{rate=}\KeywordTok{mean}\NormalTok{(x) ) )}
  \ControlFlowTok{if}\NormalTok{(type}\OperatorTok{==}\StringTok{"Gamma"}\NormalTok{) }\KeywordTok{return}\NormalTok{(}\KeywordTok{MLE_Gamma}\NormalTok{(x))}
  \ControlFlowTok{if}\NormalTok{(type}\OperatorTok{==}\StringTok{"Beta"}\NormalTok{) }\KeywordTok{return}\NormalTok{(}\KeywordTok{MLE_Beta}\NormalTok{(x))}
  \CommentTok{#if(type=="t") return(2*(a1^2+a2)/(a1^2+a2-1))}
  \CommentTok{#if(type=="chi") return(2* )}
\NormalTok{\}}
\end{Highlighting}
\end{Shaded}

\hypertarget{test}{%
\section{Test}\label{test}}

We combine parametric bootstrapping and KS-test to test the goodness of
fit.

\hypertarget{parametric-bootstrapping}{%
\subsubsection{Parametric
Bootstrapping}\label{parametric-bootstrapping}}

In parametric bootstrapping, we sample from our ``estimated''
distribution \(f(x; \hat{\theta}_n)\) instead of the empirical
distribution \(\hat{F}_n(x)\)

\hypertarget{kolmogorov-smirnov-test}{%
\subsubsection{Kolmogorov-Smirnov test}\label{kolmogorov-smirnov-test}}

In statistics, the Kolmogorov-Smirnov test (K-S test or KS test) is a
nonparametric test of the equality of continuous, one-dimensional
probability distributions that can be used to compare a sample with a
reference probability distribution (one-sample K-S test), or to compare
two samples (two-sample K-S test). It is named after Andrey Kolmogorov
and Nikolai Smirnov.

The Kolmogorov-Smirnov statistic quantifies a distance between the
empirical distribution function of the sample and the cumulative
distribution function of the reference distribution, or between the
empirical distribution functions of two samples. The null distribution
of this statistic is calculated under the null hypothesis that the
sample is drawn from the reference distribution (in the one-sample case)
or that the samples are drawn from the same distribution (in the
two-sample case). In each case, the distributions considered under the
null hypothesis are continuous distributions but are otherwise
unrestricted.

The two-sample K-S test is one of the most useful and general
nonparametric methods for comparing two samples, as it is sensitive to
differences in both location and shape of the empirical cumulative
distribution functions of the two
samples.(\url{https://en.wikipedia.org/wiki/Kolmogorov\%E2\%80\%93Smirnov_test})

\hypertarget{bootstrap-and-ks-test-code}{%
\section{Bootstrap and KS-test code}\label{bootstrap-and-ks-test-code}}

I created two \emph{partial} functions for later use, which can create a
parameters-partially-fed function.

\begin{Shaded}
\begin{Highlighting}[]
\NormalTok{partial <-}\StringTok{ }\ControlFlowTok{function}\NormalTok{(f, ...) \{}
\NormalTok{  l <-}\StringTok{ }\KeywordTok{list}\NormalTok{(...)}
  \ControlFlowTok{function}\NormalTok{(...) \{}
    \KeywordTok{do.call}\NormalTok{(f, }\KeywordTok{c}\NormalTok{(l, }\KeywordTok{list}\NormalTok{(...)))}
\NormalTok{  \}}
\NormalTok{\}}
\NormalTok{partial_list<-}\ControlFlowTok{function}\NormalTok{(f, L)\{}
  \ControlFlowTok{function}\NormalTok{(...)\{}
    \KeywordTok{do.call}\NormalTok{(f, }\KeywordTok{c}\NormalTok{(L, }\KeywordTok{list}\NormalTok{(...)))}
\NormalTok{  \}}
\NormalTok{\}}


\NormalTok{BootKS<-}\ControlFlowTok{function}\NormalTok{(x, n0 , }\DataTypeTok{nboot=}\DecValTok{10000}\NormalTok{, rvecFUN, Distritype, FUN)}
\NormalTok{\{}
\NormalTok{Dvec<-}\OtherTok{NULL}
\ControlFlowTok{for}\NormalTok{(i }\ControlFlowTok{in} \DecValTok{1}\OperatorTok{:}\NormalTok{nboot)\{}
\NormalTok{  vec<-}\KeywordTok{rvecFUN}\NormalTok{(}\DataTypeTok{n=}\NormalTok{n0)}
\NormalTok{  theta<-}\KeywordTok{MLE}\NormalTok{(vec, }\DataTypeTok{type =}\NormalTok{ Distritype)}
\NormalTok{  Dvec<-}\KeywordTok{c}\NormalTok{(Dvec, }\KeywordTok{unname}\NormalTok{(}\KeywordTok{ks.test}\NormalTok{(x, }\KeywordTok{partial_list}\NormalTok{(FUN, theta))}\OperatorTok{$}\NormalTok{statistic ))}
\NormalTok{\}}
\KeywordTok{return}\NormalTok{(Dvec)}
\NormalTok{\}}

\NormalTok{KS<-}\ControlFlowTok{function}\NormalTok{(x, }\DataTypeTok{nboot=}\DecValTok{10000}\NormalTok{, rvecFUN, Distritype, FUN)\{}
\NormalTok{  n=}\KeywordTok{length}\NormalTok{(x)}
\NormalTok{  DVec<-}\OtherTok{NULL}
\NormalTok{  theta0<-}\KeywordTok{MLE}\NormalTok{(x, }\DataTypeTok{type=}\NormalTok{Distritype)}
\NormalTok{  D0=}\KeywordTok{unname}\NormalTok{( }\KeywordTok{ks.test}\NormalTok{(x, }\KeywordTok{partial_list}\NormalTok{(FUN, theta0))}\OperatorTok{$}\NormalTok{statistic )}
   
  \ControlFlowTok{for}\NormalTok{(i }\ControlFlowTok{in} \DecValTok{1}\OperatorTok{:}\NormalTok{nboot)\{}
\NormalTok{    xstar<-}\KeywordTok{partial_list}\NormalTok{(rvecFUN, theta0)(n)}
\NormalTok{    thetastar<-}\KeywordTok{MLE}\NormalTok{(xstar, }\DataTypeTok{type =}\NormalTok{ Distritype)}
\NormalTok{    Dstar=}\KeywordTok{unname}\NormalTok{(}\KeywordTok{ks.test}\NormalTok{(xstar, }\KeywordTok{partial_list}\NormalTok{(FUN, thetastar))}\OperatorTok{$}\NormalTok{statistic )}
\NormalTok{    DVec<-}\KeywordTok{c}\NormalTok{(DVec, Dstar)}
    
\NormalTok{  \}}
  \KeywordTok{return}\NormalTok{(}\KeywordTok{list}\NormalTok{(}\DataTypeTok{p=} \KeywordTok{sum}\NormalTok{(DVec}\OperatorTok{>}\NormalTok{D0)}\OperatorTok{/}\NormalTok{nboot ))}
\NormalTok{\}}
\end{Highlighting}
\end{Shaded}

\hypertarget{test-with-normal-distribution}{%
\subsubsection{Test with Normal
Distribution}\label{test-with-normal-distribution}}

\begin{Shaded}
\begin{Highlighting}[]
\KeywordTok{set.seed}\NormalTok{(}\DecValTok{553}\NormalTok{)}
\NormalTok{x0<-}\KeywordTok{rnorm}\NormalTok{(}\DataTypeTok{n=}\DecValTok{100}\NormalTok{, }\DataTypeTok{mean=}\NormalTok{, }\DataTypeTok{sd=}\DecValTok{6}\NormalTok{)}
\KeywordTok{print}\NormalTok{( }\KeywordTok{KS}\NormalTok{(x0, }\DataTypeTok{rvecFUN =}\NormalTok{rnorm, }\DataTypeTok{Distritype =} \StringTok{"Normal"}\NormalTok{, }\DataTypeTok{FUN =}\NormalTok{ pnorm) )}
\end{Highlighting}
\end{Shaded}

\begin{verbatim}
## $p
## [1] 0.3263
\end{verbatim}

\hypertarget{test-with-gamma-distribution}{%
\subsubsection{Test with Gamma
Distribution}\label{test-with-gamma-distribution}}

\begin{Shaded}
\begin{Highlighting}[]
\KeywordTok{set.seed}\NormalTok{(}\DecValTok{239}\NormalTok{)}
\NormalTok{x1<-}\KeywordTok{rgamma}\NormalTok{(}\DataTypeTok{n=}\DecValTok{100}\NormalTok{, }\DataTypeTok{shape=}\DecValTok{5}\NormalTok{, }\DataTypeTok{scale=}\DecValTok{6}\NormalTok{)}

\KeywordTok{print}\NormalTok{(}\KeywordTok{KS}\NormalTok{(x1, }\DataTypeTok{rvecFUN =}\NormalTok{ rgamma, }\DataTypeTok{Distritype =} \StringTok{"Gamma"}\NormalTok{, }\DataTypeTok{FUN=}\NormalTok{pgamma))}
\end{Highlighting}
\end{Shaded}

\begin{verbatim}
## $p
## [1] 0.3167
\end{verbatim}

\hypertarget{test-with-beta-distribution}{%
\subsubsection{Test with Beta
Distribution}\label{test-with-beta-distribution}}

\begin{Shaded}
\begin{Highlighting}[]
\KeywordTok{set.seed}\NormalTok{(}\DecValTok{461}\NormalTok{)}
\NormalTok{x2<-}\KeywordTok{rbeta}\NormalTok{(}\DataTypeTok{n=}\DecValTok{100}\NormalTok{, }\DataTypeTok{shape1=}\DecValTok{45}\NormalTok{, }\DataTypeTok{shape2 =} \DecValTok{13}\NormalTok{)}
\KeywordTok{print}\NormalTok{( }\KeywordTok{KS}\NormalTok{(x2,}\DataTypeTok{rvecFUN =}\NormalTok{ rbeta, }\DataTypeTok{Distritype =} \StringTok{"Beta"}\NormalTok{, }\DataTypeTok{FUN =}\NormalTok{ pbeta))}
\end{Highlighting}
\end{Shaded}

\begin{verbatim}
## $p
## [1] 0.5161
\end{verbatim}


\end{document}
